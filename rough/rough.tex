\documentclass[12pt,a4paper]{report}
\usepackage[latin1]{inputenc}
\usepackage{amsmath}
\usepackage{amsfonts}
\usepackage{amssymb}
\author{Lauren Frazier}
\title{Evaluation of Gesture-Based Controls for Robotic Systems}
\begin{document}
\maketitle

\setcounter{page}{1}
\pagenumbering{roman}
\tableofcontents
\listoftables
\listoffigures

\chapter*{Acknowledgements}
\addcontentsline{toc}{chapter}{Acknowledgements}
These are acknowledgements.

\chapter*{Abstract}
\addcontentsline{toc}{chapter}{Abstract}
This thesis aims to test the effectiveness/ease of use of smartphone gesture-based robotic control systems vs. traditional control systems.  Robotic control systems are becoming more common, especially in the military. Arm and hand gestures are typical human forms of communication, so applying that to a robotic control system can yield a more intuitive system. With military applications, there are lives at stake, so having the most efficient, intuitive control system can make a large difference in the success of a mission and the safety of the soldiers involved. 
"Interactions and Training with Unmanned Systems and the Nintendo Wiimote" (Varcholik, Barber, and Nicholson) describes a gesture based control system that uses the Nintendo Wiimote to determine arm/hand gestures and control a robot. I propose to create a gesture based system using a smartphone and conduct an experiment similar to Varcholik, Barber, and Nicholson, collecting survey data from the participants on the the effectiveness and ease of use of each system.

\chapter{Introduction}
\pagestyle{headings}
\setcounter{page}{1}
\pagenumbering{arabic}

This thesis aims to test both the perceived and actual effectiveness and ease of use of smartphone gesture-based robotic control systems vs. traditional control systems.

%Robotic control systems are becoming more common, especially in the military. Arm and hand gestures are typical human forms of communication, so applying that to a robotic control system can yield a more intuitive system. With military applications, there are lives at stake, so having the most efficient, intuitive control system can make a large difference in the success of a mission and the safety of the soldiers involved.
The need for human-robot interfaces is increasing rapidly. The military has already begun using unmanned vehicles in several different arenas (air, ground, water). In order to develop the most efficient human-robot interface, we turned to a traditional form of human-human communication, arm and hand gestures. Arm and hand gestures are typical human forms of communication, so applying that to a robotic control system can yield a more intuitive system. With military applications, there are lives at stake, so having the most efficient, intuitive control system can make a large difference in critical moments and improve the safety of those involved. A more intuitive system will also reduce the training time and expenses for the operators of the vehicle.

% "Interactions and Training with Unmanned Systems and the Nintendo Wiimote" (Varcholik, Barber, and Nicholson) describes a gesture based control system that uses the Nintendo Wiimote to determine arm/hand gestures and control a robot. They then conducted a study where subjects used Wiimote gesture system and a more standard system and filled out a survey to indicate how effective the Wiimote system was as compared to the standard system.
 
%  I propose to create a gesture based system using a smartphone and conduct an experiment similar to Varcholik, Barber, and Nicholson, with one major difference: collecting survey data AND collecting data via biosensors. The system will use a Samsung Galaxy S II phone for the gesture-based input and a Microsoft XBOX 360 controller for the traditional input. Subjects will use the controls to guide a Roomba through an obstacle course. While subjects complete the experiment, their Blood Volume Pressure/Heart Rate and Galvanic Skin Response will be recorded. After the experiment is complete, subjects will also fill out a survey, and both sets of data will be used to determine which control scheme is more effective and intuitive.

%This paper makes the following contributions:
%(1)  Introduces a gesture based control system using a smartphone.
%(2)  Provides data showing physical reactions to different control systems in addition to survey results after the experiment.


\chapter{Related Work}

\chapter{Control Systems}

\chapter{Human Factors Experiment Design}

\chapter{Results}

\chapter{Conclusion}

\bibliographystyle{plain}
\bibliography{rough}

\end{document}